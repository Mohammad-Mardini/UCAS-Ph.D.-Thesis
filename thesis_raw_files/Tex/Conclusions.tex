\chapter{Conclusions}\label{chap:introduction}

In this work, we analyzed 12 metal-poor stars chosen from the LAMOST database
and followed-up, for the first time, with high-resolution observations using
Lick/APF.  We have presented stellar parameters and full detail chemical
abundances (25 individual elements from C to Eu), for these stars. Our analysis
shows no significant differences compared with the general abundance trends as
reported in previous studies of metal-poor stars. In particular, the chemical abundance patterns of our sample stars reveal that our sample contain five carbon normal stars and seven carbon-enhanced metal-poor (CEMP) stars: (i)The CEMP-r/s star J2114$-$0616 with [C/Fe]=1.37,  [N/Fe]= 1.88, [Ba/Fe]=1.00, and [Eu/Fe]=0.84, (ii) The CEMP-r star J1054+0528 with [C/Fe]=0.82, [Eu/Fe]=0.44 and [Ba/Fe]=$-$0.52 (iii) The CEMP-no ([C/Fe] $\geqslant +0.7$ and [Ba/Fe] $< 0.0$) stars J1529+0804, J1630+0953, J1645+4357, J2216+0246, and J2216+2232.. 

We have attempted to characterize the formation scenario and the progenitors of
our CEMP-no stars. Our program stars are located above the solar D$_{trans}$
values and the Forbidden Zone, suggesting that they may have likely formed from
a gas cloud exhibiting fine-structure cooling process. Furthermore, we have
compared $10^{4}$ generated sets of the determined chemical abundances of the
light-element abundances of J1630+0953 and J2216+0246 with predicted yields from
nonrotating massive-star models. About $\sim ~94\%$ of the models predicted that
the mass and explosion energy of the J1630+0953 progenitor, could be in the
21-25 M$_{\odot}$ mass range and $0.3 \times 10^{51}$ erg, respectively, and
only about $\sim ~6\%$ with mass of 50 M$_{\odot}$ and explosion energy of $10.0
\times 10^{51}$ erg.  In about $\sim ~85\%$ of the models predicted 13-16
M$_{\odot}$ mass range and explosion energy  of $1.5$-$1.8 \times 10^{51}$ erg
for the progenitor of J2216+0246, while the remaining models ($\sim ~15\%$)
predicted 10.6-29.5 M$_{\odot}$ mass range and explosion energy $0.6$-$5
\times 10^{51}$ erg. In general, our comparison suggested that massive
stellar progenitors shall be the pollutant source of their birth cloud, and
these pollutants then acted as cooling agents. Our result is consistent with
recent conclusions given by \citet{2018ApJ...857...46I}, which suggest
possible progenitors in the 15-25 M$_{\odot}$ mass range. This peak
($\sim$ 20 M$_{\odot}$) may reflect the Pop III initial mass function.
However, this brings the possibility as to whether more massive SN might be
more energetic and therefore destroy their host halo and not allow for EMP
star formation afterwards.

We have further investigated the kinematics and dynamics of the sample stars,
based on Gaia DR2 data. Our results show that all stars are members of the
inner-halo population. Nevertheless, the deficiency in iron and enhancement in
carbon abundances of J1630+0953 and J2216+0246 strongly suggest that these stars
were born in low-mass sub-galactic systems and later accreted during the initial
phases of Galaxy assembly and contributed to the old stellar populations of the
inner halo.
